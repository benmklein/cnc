\documentclass[]{article}
\usepackage{amssymb}
\usepackage[margin=0.5in,
includehead, includefoot
]{geometry}
\usepackage{amsmath}
\usepackage{qtree}
\usepackage{fancyhdr}

%opening
\pagestyle{fancy}
\fancyhf{}

\title{Week 1: Types}
\author{Ben Klein}
\lhead{Ben Klein}
\chead{Week 1: Types}
\begin{document}

\maketitle

\textbf{}\\


\begin{flushleft}
\begin{enumerate}
	\item
	\begin{enumerate}
		\item To prove these functions are extensionally equal, we must prove they produce the same output $f(n)$ for all $n$
		\item Base case: $n=1$\\
		\begin{align*}
				\sum_{i=1}^{1} i = \frac{1(1+1)}{2}\ =1
		\end{align*}
		Inductive hypothesis: assume
		\begin{align*}
			\sum_{i=1}^{k} i = \frac{k(k+1)}{2}\
		\end{align*}
		Recursive step: Prove that\\
		\begin{align*}
			\sum_{i=1}^{n+1} i = \frac{(n+1)((n+1)+1)}{2}\
		\end{align*}
		Separating the summation term:
		\begin{align*}
			\sum_{i=1}^{n+1} i &= \sum_{i=1}^{n} i\ + (n+1) \\\\
			\text{Using the inductive hypothesis:}\\
			&= \frac{n(n+1)}{2}\ + (n+1)\\
			&= \frac{n(n+1) + 2(n+1)}{2}\\
			&= \frac{(n+1)((n+1)+1)}{2}\
		\end{align*}
	\end{enumerate}
\item 
\begin{enumerate}
	\item To prove that strong induction is equivalent to the principle of induction you must prove they are extensionally equal.
\end{enumerate}
\item Pierce 2.2.6\\
$R^\prime = R \cup\ \{(s,s)|s\epsilon S \}$
The definition of reflexive closure $R^{\prime \prime}$ is the smallest reflexive relation $R$ on $s\epsilon S$.
In other words $R^{\prime \prime} = R \cup\ \{(s,s)|s\epsilon S \}$ Since $R^{\prime \prime} = R^{\prime}$, $R^{\prime}$ is the smallest possible relation that fits the definition.
\item Pierce 2.2.7\\
Using induction to prove the rules given will produce a transitive closure on R:\\
Base case: a set $R_0 = \{s,t,u,(s,t),(s,u)\}$\\
after one recursive step, $R_1$:\\
$(s,u) \epsilon R_1, R_0 \cup R_1 = $\\
$R^{+} = \{s,t,u,(s,t),(t,u),(s,u)\}$\\
This is the smallest transitive relation R.\\
Inductive hypothesis:
\begin{align*}
	R^+ \text{is the smallest transitive relation of } \bigcup_{i}{R_i}
\end{align*}
Recursive case: prove that
\begin{align*}
	\bigcup_{i}{R_{i+1}} = \bigcup_{i}{R_i}
\end{align*}
Because $\bigcup_{i}{R_i}$ is the smallest transitive closure, adding any transitive relation to the set will redundant because $\bigcup_{i}{R_i}$ contains all pairs that can be obtained from 1 step of transitivity. It is therefore the transitive closure.

\item Pierce 2.2.8\\
$R^*$ is the smallest reflexive and transitive closure on a set S. A predicate, $P$ is defined by a one place relation on a set $S$ if $s \epsilon S$, and $s \epsilon P$. If P(S) is true after the binary relation R, then to check if P is preserved by R*, we must check if P(s=s) and P(s=u) is true. (By definition of reflexive and transitive.) Since these are equivalence relations on the parameter of P, they must be true. $(s=s)\epsilon P, (s=u)\epsilon P$
\end{enumerate}
\end{flushleft}


\end{document}
