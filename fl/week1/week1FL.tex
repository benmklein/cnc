\documentclass[]{article}
\usepackage{amssymb}
\usepackage[margin=0.5in,
includehead, includefoot
]{geometry}
\usepackage{amsmath}
\usepackage{qtree}
\usepackage{fancyhdr}

%opening
\pagestyle{fancy}
\fancyhf{}

\title{Week 1: Formal Languages}
\author{Ben Klein}
\lhead{Ben Klein}
\chead{Week 1: Formal Languages}
\begin{document}

\maketitle

\textbf{Chapter 1: 1, 4, 6, 10, 22, 29, 34, 38, 42, 46
}\\

Worked with: Danny Denniston, Eric Mock and Anna Bobcova
\\
\\
\begin{flushleft}
\begin{enumerate}

	\item [\textbf{1.}]
	\begin{enumerate}
		\item \{0,1,2,3,4,6\}
		\item \{2,4\}
		\item \{1,3\} 
		\item \{0,6\}
		\item \{$\varnothing$, \{1\},\{2\},\{3\},\{4\},\{1,2\},\{2,3\},\{3,4\},\{2,4\},\{1,3\},\{1,4\},
		\{1,2,3\},\{1,3,4\},\{1,2,4\},\{2,3,4\},\{1,2,3,4\}\}
	\end{enumerate}
	
	
	\item [\textbf{4.}] 
	Proof that:
\begin{align*}
	X&= \{n^3+3n^2+3n\ |\ n \ge 0\}\\
	&=\\
	Y&= \{n^3-1\ |\ n > 0\}\\
\end{align*}
Let $a \epsilon Y$, then $a=n^3-1$, and $n>0$, and $n\epsilon \mathbb{N}$
\begin{align*}
	a&=n^3-1
\end{align*}
Let $n = $ a natural number greater than 0 ($m+1$)\\
\begin{align*}
	a&=(m+1)^3 - 1 \\
	&= m^3 + 3m^2 + 3m + 1 - 1\\
	&= m^3 + 3m^2 + 3m
\end{align*}
So if $n > 0$, and $a = n^3+3n^2+3n$, then $a\epsilon X$, and $Y\subseteq X$.\\
\bigskip
Now we prove the opposite:\\
Let $b\epsilon X$, then $b=n^3+3n^2+3n$, and $n\ge 0$
\begin{align*}
	b&=n^3+3n^2+3n\\
	&=n^3+3n^2+3n+1-1\\
	&=(n+1)^3-1
\end{align*}
Therefore: b is the cube of a natural number $>0 -1$ ($n^3-1 | n > 0$) and $b\epsilon Y$ and $X\subseteq Y$\\
Since $X\subseteq Y$ and $Y\subseteq X$: $X=Y$
\newpage
\item [\textbf{6.}]
	\begin{enumerate}
		\item $f(x) = 2x$
		\item $f(x) =  \left\lfloor\frac{1}{2}x\right\rfloor$
		\item $f(x) = x+1$ if $x\%2=0$, else if $x\%2=1$, $x-1$
		\item $f(x) = \left\lceil\frac{1}{x}\right\rceil -1$
	\end{enumerate}
\bigskip


\item [\textbf{10.}] Binary relations are equivalence relations if they satisfy reflexivity, symmetry, and transitivity.\\
\bigskip
Let $n=m$, and $n,m\epsilon \mathbb{N}$\\
\bigskip
Reflexivity:
\begin{align*}
n&=n\\
m&=m
\end{align*}
Reflexivity is true.\\
Symmetry:
\begin{align*}
\text{if } n=m, \text{then }m=n
\end{align*}
Symmetry is true.\\
Transitivity:
\begin{align*}
\text{if } n=a \text{ and } a=m \text{ then } n=m
\end{align*}
Transitivity is true.\\
$\equiv$ is therefore an equivalence relation.\\
The equivalence classes are:
\begin{align*}
	[n]_{\equiv}  &= \{m\epsilon\mathbb{N}|n\equiv m\}\\
	[m]_{\equiv}  &= \{n\epsilon\mathbb{N}|m\equiv n\}
\end{align*}

\item [\textbf{22.}] First, assume the monotone increasing functions from N x N are countable. Let $f_{n}(x)$ be all monotone increasing functions from N x N. \\
\medskip
Now consider the monotone increasing function: $g(n) = f_n(n)+1$. This must be in the list of all monotone increasing functions by definition of the list, however the function cannot match anything in the list as they differ when $f_n(n)$. Since there is a contradiction, the monotone increasing functions cannot be countable.

\item[\textbf{29.}]Basis: if $n=0$, $m=0$ then $n=m$\\
Recursive Case: if $m = n$, then $s(n) = s(m)$\\
Closure: $n,m \epsilon \mathbb{N}$\\

\item[\textbf{34.}] Basis: if $n=0$, $m=1$ then $n$ is $pred(m)$\\
Recursive Case: if $m = s(n)$, then $n =pred(m)$\\
Closure: $n,m \epsilon \mathbb{N}$\\
\newpage

\item[\textbf{38.}]Basis: when n = 1
\begin{align*}
\sum_{i=1}^{1} 3i-1 = 2 = \frac{1(3(1)+1)}{2}\
\end{align*}
Inductive Hypothesis: for all $k=1,2...n$:
\begin{align*}
	\sum_{i=1}^{k} 3i-1 = \frac{k(3k+1)}{2}\
\end{align*}
Inductive step: the goal is to prove using the inductive hypothesis that
\begin{align*}
	\sum_{i=1}^{n+1} 3i-1 &= \frac{(n+1)(3(n+1)+1)}{2}\ \\&= \frac{(n+1)(3n+4)}{2}\ \\
\end{align*}
First, break the summation into 2 parts:
\begin{align*}
	\sum_{i=1}^{n+1} 3i-1 &= \sum_{i=1}^{n} 3i-1 + (3(n+1)-1)\\\
\end{align*}
then using the inductive hypothesis:
\begin{align*}
	&= \frac{(n)(3(n)+1)}{2} + (3(n+1)-1)\\
	&= \frac{3n^2 + n}{2} + 3n + 2\\
	&= \frac{3n^2 + n+6n+4}{2}\\
	&= \frac{3n^2 + 7n +4}{2}\\
	&= \frac{(n+1)(3n+4)}{2}
\end{align*}
This proves the inductive hypothesis is true.

\item[\textbf{42.}]
\begin{enumerate}
	\item $E_0 = \{A,B\}$\\
	$E_1 = \{(A\wedge B),(A\vee B)\}$\\
	$E_2 = \{((A\wedge B)\vee (A\vee B)),((A\wedge B)\wedge (A\vee B)),((A\wedge B)\wedge A)((A\wedge B)\vee A),((A\vee B)\wedge A)((A\vee B)\vee A),$\\
	$((A\wedge B)\wedge B)((A\wedge B)\vee B),((A\vee B)\wedge B),((A\vee B)\vee B) \}$\\
	\item Basis: 
	\begin{align*}
	n_p((u\vee v)) &= n_p((u\wedge v))\\
	n_p((u\vee v)) &= 2\\
	n_o((u\vee v)) &= 1\\
	\end{align*}
	So, in the base case, $n_p(u) = n_o(u) +1$\\
	Inductive hypothesis: assume for all expressions $k = u_i, i = 1,2...$
	\begin{align*}
	n_p(k) = n_o(k)+1
	\end{align*}
	Inductive step: prove that
	\begin{align*}
	n_p((u\vee v)\vee w)) = n_o((u\vee v)\vee w)) + 1
	\end{align*}
	for all $w$. In the recursive step, new terms are added to $E$ by adding one proposition letter and one letter. So,
	\begin{align*}
	n_p((u\vee v) + 1 = n_o((u\vee v) +1  + 1
	\end{align*}
	It follows that with each application of the recursive step, the number of terms will be 1 greater than the number of operators.
	
	\item Base case: $i=0$\\
	$n_L(u),n_R(u)$ means number of right and left parenthesis in an expression $u$\\
	$n_L(E_0) = 0 = n_R(E_0)$
	We see in the base case: $\{A,B\}$, there are no parentheses therefore the number of parentheses is the same.\\
	\bigskip
	Hypothesis:	$k = 1,2...$\\
	\begin{align*}
	n_L(E_k) = n_R(E_k)
	\end{align*}
	Prove:
	\begin{align*}
		n_L(E_{i+1}) = n_R(E_{i+1})
	\end{align*}
	There are 2 cases when applying the recursive step, either you add 1 left parenthesis and 1 right parenthesis by forming $(u\vee v)$ or you create $(u\wedge v)$ and also add 1 parenthesis on both sides. So,
	\begin{align*}
	n_L(E_{i+1}) = n_L(E_{i})+1=n_R(E_{i+1}) = n_R(E_{i})+1
	\end{align*}
	Therefore the number of parentheses will remain the same.
	
\end{enumerate}

\item [\textbf{46.}]
\begin{enumerate}
	\item 4
	\item $x_{11}, x_{7}, x_{2}, x_{1}$
	\item $x_2, x_1$
	\item 
	\Tree[.x_2 [.x_5 [.x_{10} .x_{14} ]]
	[.x_6 ]
	[.x_7 x_{11} ]]
\item $x_{14}, x_{6}, x_{11}, x_{3}, x_{8}, x_{12}, x_{15}, x_{16}$
\end{enumerate}
\end{enumerate}
\end{flushleft}


\end{document}
